\documentclass{article}
%!TEX root =  main.tex

% packages
\usepackage{graphicx}
\usepackage{xcolor}
\newcommand\todo[1]{\textcolor{red}{#1}}
\usepackage{amsfonts, amsmath} 

\newtheorem{prodefinition}{Problem Statement}

\usepackage{tabularx}
\usepackage{caption}
\captionsetup[table]{skip=10pt}
\usepackage{cleveref}
\newtheorem{Example}{Example}[section]

\usepackage{listings}
\usepackage[nomargin,inline,draft]{fixme}
\usepackage{xcolor,colortbl}
\usepackage{ADT_web}
\usepackage{algorithm2e}
\usepackage{algorithmic}
\usepackage{listings}
\usepackage{mathtools}
\usepackage{mdframed}
\usepackage{siunitx}
\lstset{
	basicstyle=\ttfamily,
	mathescape
}

\fxusetheme{color}
\fxsetup{inlineface=\sffamily\slshape}
\fxsetup{envface=\sffamily\slshape}
\FXRegisterAuthor{pm}{pme}{PM}

% new command
\newcommand{\adand}{\land_\textsf{AD}}
\newcommand{\ador}{\lor_\textsf{AD}}
\newcommand{\attop}{\textsf{Attack}}

\newcommand{\dand}{\land_\textsf{D}}
\newcommand{\dor}{\lor_\textsf{D}}
\newcommand{\defop}{\textsf{Defense}}

\newcommand{\emtt}[1]{\emph{\texttt{#1}}}
\newcommand{\ba}{\texttt{b}^\texttt{A}}
\newcommand{\ora}{\texttt{OR}^\texttt{A}}
\newcommand{\anda}{\texttt{AND}^\texttt{A}}
\newcommand{\ca}{\texttt{C}^\texttt{A}}
\newcommand{\cd}{\texttt{C}^\texttt{D}}
\newcommand{\ord}{\texttt{OR}^\texttt{D}}
\newcommand{\andd}{\texttt{AND}^\texttt{D}}
\newcommand{\bd}{\texttt{b}^\texttt{D}}
\newcommand{\cost}{\mathbb{C}}
\newcommand{\model}{\mathbb{M}}
\newcommand{\imodel}{\mathbb{M}_{\mathbb{I}}}
\newcommand{\amodel}{\mathbb{M}_{\mathbb{A}}}
\newcommand{\smodel}{\mathbb{M}_{\mathbb{S}}}
\newcommand{\cursor}{\texttt{cursor}}
\newcommand{\trace}{\texttt{trace}}
\newcommand{\child}[1]{\texttt{child#1}}
\newcommand{\rootn}{\texttt{root}}

\newcommand{\arjun}[1]{\textcolor{red}{#1}}

\title{Internship Report from Summer 2021 at GE Research}
\author{Arjun Viswanathan}

\date{\today}
\begin{document}

\maketitle

\section{Cost Models}
	\label{sec:costmodel}
	MBAS (model-based architectural synthesis) 
	looks at the following:
	\begin{enumerate}
		\item AADL model of a system 
		\item Cyber requirements
		\item Possible attacks 
		\item Possible defenses 
		\item Implemented defenses 
		\item Cost model for defenses
	\end{enumerate}
	and is able to do the following:
	\begin{enumerate}
		\item Determine whether the cyber
		requirements are satisfied by the
		implemented defenses, ie, if the 
		attacks are successfully mitigated.
		\item Suggest a minimal-cost defense 
		strategy, that would satisfy  the 
		requirements. 
	\end{enumerate}
	The cost model specifies the cost of 
	implementing a defense. The only 
	restriction placed on the cost model
	is that the cost model is monotonically
	increasing with the DAL score for a 
	component-defense pair --- for DAL scores
	$\delta_i$ and $\delta_j$ such that 
	$\delta_i < \delta_j$, the cost of 
	implementing defense $d$ in component
	$s$ to $\delta_i$ is less than or 
	equal to the cost of implementing it 
	to $delta_j$. The following work was
	done regarding cost models:
	\begin{enumerate}
		\item The current cost model for the 
		delivery drone system assigned the cost 
		of implementing any defense to some 
		DAL score $\delta$ as just $\delta$.
		We came up with a more realistic cost 
		model with more interesting costs for 
		the component-defense pairs after
		doing some research about the 
		components and the defenses.
		\item We introduced a methodology to 
		come up with these costs by separating 
		costs into aspects such as effort and 
		time and aggregating these scores.
		\item These values are entered into 
		a CSV file, and we created a 
		translator from the CSV file to the 
		XML format of the cost model input to
		synthesis.
		\item Even though we have tried to 
		reason about the right costs to 
		be assigned to defenses, since a 
		security expert might do a better 
		job of this, we have generalized 
		this process enough, where the security
		expert need only fill in the values
		in the CSV, and our translator would
		create the cost model from these
		values.
	\end{enumerate}

\section{Formalizing Synthesis}
	The current Synthesis implementation, as
	described in~\ref{sec:costmodel} hadn't 
	been clearly formalized in a paper. There 
	were 3 highly incomplete papers, which 
	didn't fully describe how Synthesis 
	worked. We accomplished the following goals
	to the end of formalizing the current 
	Synthesis implementation:
	\begin{enumerate}
		\item We described our attack-defense
		trees (ADTrees) using their formalism 
		in the security literature.
		This allows us to use all the advances
		made in the area of ADTrees
		and also clearly describe what we are
		able to achieve, and what we are not.
		\item We described the algorithm used 
		to convert an AADL component model to 
		an ADTree. This entailed formalizing 
		the (CSV) output given by STEM (as a 
		relation).
		\item We describe how we evaluate these 
		ADTrees to see whether 
		they satisfy the requirements, as an 
		inductive function.
		\item We describe how we encode ADTrees
		as MaxSMT queries. 
		\item We describe how we interpret the 
		response from the SMT solver as a 
		defense implementation.
		\item We came up with a running example of 
		an house alarm system to demonstrate the 
		point to a generic audience.
		\item We also added the example of a 
		simple drone system to keep it 
		domain-specific.
	\end{enumerate}

\section{Synthesis Implementation}
	While formalizing the Synthesis implementation,
	we differentiated two instances of the 
	problem --- Synthesis with and without 
	dependent attacks. A dependent attack is 
	one which is dependent on the implementation 
	of a defense. In the dependent attacks, 
	Synthesis would need to propose defenses for 
	possible attacks, plus defenses for attacks 
	dependent on these defenses. And it should 
	be able to do this to a fixpoint. According
	to the current STEM rules, this is only 
	possible up to one iteration. That is, none
	of the defenses that mitigate dependent 
	attacks have attacks dependent on them. In 
	fact, only two current defenses have dependent
	attacks. The following was achieved under the 
	implementation front.
	\begin{enumerate}
		\item We identified the limits of the 
		current dependent attacks as mentioned 
		above.
		\item It wasn't clear whether dependent 
		attacks were currently implemented or not.
		The incomplete papers mentioned them, the 
		code had mentions of them, but running 
		Synthesis showed that it was taking
		multiple iterations to work sometimes
		(with dependent attacks, it never would).
		So we did a combination of exploring the
		code, commenting it, and testing the current 
		implementation. We found that dependent 
		attacks, because of the small number of 
		them, were hard-coded into the implementation.
		That is, the ADTrees in the implementation
		couldn't handle generic dependent attacks,
		instead, they could deal only with the 
		particular ones that were described by 
		the STEM rules. We also found a bug 
		in the hard-coding of one of the attacks, 
		which is why the implementation took 
		multiple iterations on some examples.
		We fixed the bug, and now the current
		Synthesis implementation works for the 
		current list of dependent attacks.
		\item We tested synthesis for its merit 
		assignment functionality and for its 
		ability to choose cheaper defenses. 
		\begin{itemize}
			\item DeliveryDroneCase2a demonstrates 
			merit assignment or Case 2(a) from
			the problem statement in paper. The 
			\texttt{fc} and \texttt{positionEstimator}
			components of the delivery drone can 
			implement either physicalAccessControl (PAC)
			or systemAccessControl (SAC). We implement 
			PAC in both components, and create a cost 
			model where it costs 700 units to implement 
			PAC to DAL 7 in both components, 
			whereas it only costs 7 units to implement
			SAC in both components. But because SAC is 
			already implemented, if we use implemented
			defenses while running synthesis, it will 
			suggest reducing the DAL of SAC to 7, 
			but not to remove it and replace it with PAC. 
			\item DeliveryDroneCostModel illustrates
			a simple case where Synthesis would pick 
			the cheaper alternative from different
			possible defenses that would work. Here,
			neither SAC nor PAC is implemented for 
			\texttt{fc} and \texttt{positionEstimator}.
			If we make one of them more expensive than
			the other in the cost model, Synthesis will
			pick the cheaper one to implement. The 
			two test case cost models are commented 
			out in the cost model file.
		\end{itemize}
		\item We noted the difference between 
		the implementation and the formalism, 
		which might be an important thing to 
		defend in top conferences. This difference
		is elaborated in the following subsection.
	\end{enumerate}

	\subsection{ADTrees in Formalism vs Implementation}
	We stick to recursive functions on regular 
	ADTrees in the formalism for previously
	mentioned reasons. 
	\begin{figure}[t]
		\scalebox{0.75}{\pstree[levelsep=2.5cm,treesep=0.4cm]
		{\NodeA{$\anda$}}{
		\pstree[levelsep=2.5cm,treesep=0.4cm]
		{\NodeA{$\ba$\\(CAPEC\\137)}}{
			\pstree[levelsep=2.5cm,treesep=0.4cm]
			{\NodeBC{$\bd$(Dev\\Auth)}}{
				\pstree[levelsep=2.5cm,treesep=0.4cm]
				{\NodeAC{$\ba$\\(CAPEC\\114)}}{
					\NodeBC{$\bd$(Static\\Code\\Analysis)}
				}
			}
		}}}
		\hspace{1cm}
		\scalebox{0.75}{\pstree[levelsep=2.5cm,treesep=0.4cm]
		{\NodeA{\texttt{ADOr}}}{
			\pstree[levelsep=2.5cm,treesep=0.4cm]
			{\NNodeA{n1}{\texttt{ADAnd}}}{
				\pstree[levelsep=2.5cm,treesep=0.4cm]
				{\NNodeA{n2}{\texttt{ADNot}}}{
					\NodeBC{Static\\Code\\Analysis}
				}
				\NodeA{CAPEC\\114}
				\pstree[levelsep=2.5cm,treesep=0.4cm]
				{\NNodeA{n3}{\texttt{ADOr}}}{
					\NodeBC{\texttt{DC}:\\c.Atkble,\\DevAuth,\\ 1}
				}
			}
			\pstree[levelsep=2.5cm,treesep=0.4cm]
			{\NNodeA{n4}{\texttt{ADAnd}}}{
				\pstree[levelsep=2.5cm,treesep=0.4cm]
				{\NNodeA{n5}{\texttt{ADNot}}}{
					\NodeBC{Dev\\Auth}}
				\NNodeA{n6}{CAPEC\\137}
			}
		}
		\Arc{n1}{n2}{n3}
		\Arc{n4}{n5}{n6}}
		\caption{Difference between current formalism of 
		dependent attacks (left) and current 
		implementation (on the right) for a particular
		instance. Dev Auth is Device Authentication, 
		CAPEC-137 is Parameter Injection, CAPEC-114 is
		Authentication Abuse, DC stands for 
		DefenseCondition, and c.Atkble represent 
		an \textit{attackable} of the incoming
		connection that encapsulate all possible
		attacks and defenses of it.
		}
		\label{fig:dep}
	\end{figure}
	The ADTree in the 
	implementation is slightly different.
	An ADTree, is split into the following 
	classes, all defined in 
	\texttt{attackdefensecollector/adtree/}:
	\begin{itemize}
		\item \texttt{ADAnd} corresponds to
		 $\texttt{AND}^{\texttt{A}}$ nodes.
		\item \texttt{ADOr} corresponds to
		$\texttt{OR}^{\texttt{A}}$ nodes.
		\item \texttt{Attack} corresponds to 
		$\texttt{b}^{\texttt{A}}$ nodes.
		\item \texttt{Defense} corresponds to an
		entire defense tree.
		\item \texttt{ADNot} is used to represent 
		Defense nodes as counteracting 
		attack-nodes. A 
		$\texttt{C}^{\texttt{A}}(a, D)$ node 
		from our formalism corresponds 
		to ADAnd(a, ADNot(D)).
		\item \texttt{DefenseCondition} is a special
		node representing dependent attacks.
		It encapsulates the defense on 
		which the attack is dependent, 
		the implemented DAL, and the attacks
		and defenses applicable to the 
		incoming connection; it doesn't hold 
		the dependent attack itself. Instead, 
		it is And-ed with the dependent attack.
	\end{itemize}
	Figure~\ref{fig:dep} shows an instance of 
	an attack, its defense, an attack
	dependent on the defense, and finally, 
	the defense against the dependent attack
	represented using formalism on the left, 
	and using the current implementation on 
	the right.
	The following table compares the 
	implementation to the formalism and can be 
	used as developer documentation to understand
	the current implementation:
	\begin{center}
		\begin{tabular}{| p{3cm} | p{5cm} | p{5cm} |}
		\hline
		\textbf{Text} & 
		\textbf{Implementation} &
		\textbf{Formalism}\\
		\hline
		Entry Point & 
		\texttt{runMbasSynthesis()} in
		\texttt{verdict-bundle/App.java/} & \\
		\hline
		Defense Model & 
		\texttt{Defense.csv} output by STEM & 
		3.1 Defense Models \\
		\hline
		AADL Model $\to$ ADTree & 
		\texttt{Trace()} in 
		\texttt{SystemModel.java}, called 
		from \texttt{perform()} in
		\texttt{AttackDefenseCollector.java} &
		\texttt{ADTree} algorithm in
		Section 3.2 ADTree Construction \\
		\hline
		ADTree Evaluation & 
		\texttt{compute()} in 
		\texttt{ADTree.java}, called 
		from \texttt{perform()} in
		\texttt{AttackDefenseCollector.java} & 
		$M()$ in Section 3.3
		ADTree Evaluation\\
		\hline 
		Cost Model & 
		Input as \texttt{costModel.xml} & 
		Defined in in Section 4 ADTree
		Synthesis \\
		\hline
		Global Synthesis Solution & 
		\texttt{partialSolution = False, 
			performMeritSolution = False} & 
		Case 1 in 4.1 Problem Statement \\
		\hline
		Use Implemented Defenses that are 
		Satisfying & 
		\texttt{partialSolution = True, 
			performMeritSolution = True} &
		Case 2(a) in 4.1 Problem Statement \\
		\hline
		Use Implemented Defenses that are 
		Not Satisfying & 
		\texttt{partialSolution = True, 
			performMeritSolution = False} &
		Case 2(b) in 4.1 Problem Statement \\
		\hline
		ADTree $\to$ DTree & 
		\texttt{construct()} in 
		\texttt{DTreeConstructor.java} & \\ 
		\hline
		ADTree/DTree $\to$ SMT query & 
		\texttt{performSynthesisMultiple()} in
		\texttt{VerdictSynthesis.java} &
		$F()$ in Section 4.2 MaxSMT Encoding \\
		\hline
		SMT Model Evaluation & 
		\texttt{performSynthesisMultiple()} in
		\texttt{VerdictSynthesis.java} & 
		Section 5 SMT Model Evaluation \\
		\hline
		Dependent Attacks & 
		Hard-coded in 
		\texttt{DependentRules.java}, 
		and checked in \texttt{trace()}
		before building the ADTree & 
		$\cd$ in the ADTree allows for any 
		generic set of dependent attacks \\
		\hline
		\end{tabular}
	\end{center}

\section{Future Work}
	\begin{enumerate}
		\item Hard-coding of dependent attacks is 
		extremely limiting. The formalism specifies 
		ADTrees as being generic enough to accommodate 
		any dependent attacks, while we are only able
		to deal with a small set of hard-coded 
		ones. From the user's point of view, if 
		there were to be a new attack that was 
		dependent on a defense, they would have to
		add to the hard-coded rules, and this is 
		very undesirable. If true dependent
		attacks were to be implemented, STEM 
		would have to change its \texttt{Defense.csv}
		output. It would have to have an extra 
		dependent attack column. Currently, each 
		row specifies a component, its applicable
		attack, and implemented and applicable 
		defenses. If the applicable attack was 
		dependent on a defense, the dependent 
		attack column would be populated with
		that defense, and it would be empty
		otherwise.
		\item Is MaxSAT good enough to encode our 
		problem? If it is, we cannot justify using 
		MaxSMT which is more expressive and has 
		more computational limitations. Researchers
		would have an issue with this, both from 
		the perspective of conference submissions, 
		and selling the product to formal 
		methods-savvy clients.
	\end{enumerate}

\end{document}